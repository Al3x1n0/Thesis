\markboth{}{Ringraziamenti}
\chapter*{Ringraziamenti}%
Voglio partire con il ringraziare il mio Relatore, il Prof. Francesco Pasquale, \`e stato sempre disponibile ogni qual volta avessi bisogno di un supporto tecnico o morale; anche in questo periodo di emergenza costretti a vederci soltanto via web (era una delle poche persone oltre la mia famiglia che vedevo). La sua cordialit\`a, precisione e competenza hanno fatto si che mi affidassi a lei per completare il mio percorso di studi nel migliore dei modi, la conferma ce l'ho avuta vedendo il lavoro di Tesi appagante che ne \`e venuto fuori (non che mi servisse una conferma).\\
Continuo con il ringraziare tutti i miei colleghi di corso che hanno condiviso con me questa splendida avventura, in particolar modo ne cito qualcuno:\\

\begin{itemize}
    \item Il brother, ho condiviso con te sin dal primo anno tutto, a partire dagli studi fino ad arrivare ai problemi della vita e i momenti di debolezza, ci siamo sempre fatti forza a vicenda ed \`e anche grazie al tuo supporto se ora sono arrivato fin qui, raggiungendo questo traguardo.
    \item Il gruppo JDMD, composto dal brother che ho citato sopra, da Ion e Lorenzo. Fantastici, mi avete accompagnato nell'esame di Ingegneria del Software, producendo un progetto bellissimo e degno di nota. Oltre a questo, ricordo con gioia i cappuccini e cornetti mangiati con te Ion prima di entrare a fare lezione di Algoritmi.
    \item Il gruppo di organizzatori dell'Hackathon 2019, un evento che mi ha portato gioia e gratificazione, Simone, Samir, Fede, Marcello e Manuel.
    \item Ringrazio poi Alessio, che se non fosse stato per te ero ancora fermo in uno degli scogli del mio corso di laurea, il tuo aiuto \`e stato essenziale, e ricorder\`o sempre la bella persona che sei.
    \item Tutto il Lab25A sempre presente per qualsiasi cosa, ragazzi fantastici che sono un punto di riferimento per il nostro corso di laurea in Informatica.
    \item Potrei mettere tra i miei colleghi universitari anche te, la mia splendida Donna che ho incontrato durante questo percorso, ma in quanto tale ti meriti uno spazio riservato $<$3, pertanto ti ringrazier\`o successivamente.
\end{itemize}

Voglio poi continuare con il ringraziare tutti i miei amici, che mi hanno incoraggiato sin dall'inizio per portare a termine questo percorso intrapreso, anche in questo caso ne cito qualcuno:

\begin{itemize}
    \item Raffo, quel giorno, in quel parcheggio.. incontrare un amico di vecchia data e scoprire che anche tu come me avevi deciso, un po' in ritardo di intraprendere questo percorso formativo offerto dall'Universit\`a \`e stata una gioia. Abbiamo condiviso molti momenti insieme e te ne sono grato, perch\'e ogni momento \`e stato tempo ben speso.
    \item Davidello, una delle persone con cui ho condiviso moltissimo tempo soprattutto durante il periodo che precedeva la sessione, quando le lezioni erano terminate e si doveva ripassare, tu sei stato sempre presente mi hai fatto compagnia nello studio. Ma oltre l'Universit\`a ti ringrazio di essermi sempre vicino anche nella vita.
    \item Come dimenticare Ganega, \`e soltanto grazie a te che ingranai la marcia giusta, tu mi diedi l'aiuto necessario per superare l'esame di Analisi Matematica, primo grande scoglio. \`E stato un piacere farmi spiegare le cose da te, una persona preparata e competente che mi ha avvantaggiato molto. Dopo aver visto gli immediati risultati positivi gi\`a al primo Appello, ebbi conferma di quanto fu utile il tuo aiuto. Anche con te i ringraziamenti non si limitano a questo ma ti ringrazio per i bei momenti passati insieme, soprattutto le giornate al poligono per scaricare la tensione esplodendo un po' di colpi.
    \item Tutto il gruppo di amici splendidi che ho sin da quando sono piccolo, non potrei chiedere di meglio, siete la fortuna pi\`u grande che questa vita mi ha regalato. Parlo di Lele, Daniele, Riccardo, Andrea, Marcello e Sergio.
    \item Ringrazio la mia sorella non di sangue, Simona, che anche se durante il mio percorso era gi\`a alla conclusione del suo (bell'amica, mica mi aspetta..) ha saputo sostenermi e consigliarmi, ma anche farmi passare belle giornate spensierate (ad esempio sulla neve.. forse li il belle vale solo per me) che sono sempre utili per staccare dallo studio.
    \item Ringrazio i miei amici Fabio, Alessandro e Chiara, che con la vostra presenza avete sempre migliorato le mie giornate!
    \item Martina, Luca, Emanuele, Danilo, ecc. i miei amici abruzzesi, avete accompagnato da sempre le mie estati in quel fantastico posto, del quale ringrazio tutti gli abitanti parenti e non.
    \item Massimo, non mi ero dimenticato di te non iniziare ad agitarti, anche a te vanno i miei ringraziamenti, mi ricordo le spiegazioni essenziali per superare l'esame di Probabilit\`a. Ma soprattutto ricordo i tantissimo momenti fuori dalla vita universitaria, hai sempre saputo farmi svagare al meglio! Approfitto per ringraziare quel tuo coinquilino pazzo che mi \`e stato simpatico sin da subito, Fede.
\end{itemize}
\`E ora arrivato il momento di ringraziare te amore mio! La mia Fatima. Quando ho cominciato l'Universit\`a non ho pensato alla possibilit\`a di trovare l'amore, anche perch\'e ho deciso di studiare Informatica, ed \`e risaputo che ad Informatica le Donne sono assai rare. Invece ho incontrato te, la donna che ha reso questo ultimo anno della mia vita fantastico, spero di avere un bel futuro e spero di averlo insieme a te. Ti voglio ringraziare per essermi stata sempre vicina, per avermi sempre supportato e sopportato, per avermi dato coraggio nei momenti bui e forza quando ne avevo bisogno.\\
Infine voglio ringraziare voi, la mia splendida famiglia, che mi ha appoggiato quando ho deciso di intraprendere questo percorso, anche se questo comportava il lasciare il lavoro per dedicarmi a tempo pieno nello studio, siete stati subito d'accordo della mia scelta, supportandomi economicamente e dandomi la possibilit\`a di formarmi e di arrivare dove sono ora. Siete la famiglia migliore che si possa desiderare, sempre disposti ad aiutare senza chiedere mai nulla in cambio, mi avete dato sempre amore e spero di ripagarvi per tutta la vita con il doppio se non il triplo dell'amore ricevuto. Voglio includere nella mia famiglia anche voi, la mia famiglia adottiva, Sonia Claudio, Luca e Simona. Grazie per esserci sempre stati.\\
Non \`e vero non ho concluso, c'\`e una persona che ho deciso di ringraziare per ultima, non voglio fare preferenze, anche perch\'e gi\`a ti ho ringraziato nella famiglia, ma volevo dedicare a te questa Laurea Padre. \`E soltanto grazie alla tua fiducia che mi hai sempre dato che sono qui ora, ho creduto in me stesso, tornando indietro ti dico che probabilmente non mi ci sarei mai visto a possedere una Laurea, e invece eccomi qui. Probabilmente la fiducia che mi hai dato e che mi dai tutt'ora \`e pi\`u della fiducia che mi do io stesso. Quindi ti ringrazio per essere come sei, non potrei pretendere un padre migliore di te. Dedico quindi a te questo traguardo e ti ringrazio per avermi sempre sostenuto, ti voglio bene.