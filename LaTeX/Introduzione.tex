\chapter*{Introduzione}
\addcontentsline{toc}{chapter}{Introduzione}

\section*{Premessa}
\addcontentsline{toc}{section}{Premessa}
Nel Capitolo \ref{background}, dove viene fornito un Background all'argomento, faccio riferimento ad enti italiani, ma il discorso sar\`a analogo per gli altri paesi, cambieranno soltanto gli enti che si occupano della raccolta e del monitoraggio dei dati e, gli enti che hanno scopi di protezione civile.\\
Quello che andr\`o a fare, \`e approcciare alla predizione dei terremoti in modo diverso dall'analisi dei precursori (vedi Sezione \ref{previsione}).\\
Quindi prender\`o in analisi soltanto i dati storici dei terremoti messi a disposizione dagli enti nazionali che li raccolgono. L'idea di fondo \`e che tutto il lavoro che andr\`o a svolgere dovr\`a essere estendibile ad ogni altra realt\`a che non sia solo quella nazionale, quindi tutto \`e mirato a sviluppare uno standard che mi permetta di analizzare una grande mole di dati a prescindere della provenienza di essa.


\section*{Scopo della tesi}
\addcontentsline{toc}{section}{Scopo della tesi}
Lo scopo di questa tesi \`e sviluppare uno o pi\`u applicativi che preso in input un catalogo contenente record\footnote{Oggetto o struttura dati eterogenei fatta da dati compositi, contenente quindi un insieme di campi o elementi, ciascuno dei quali identificato da un nome univoco e da un tipo di dato} rappresentati dai valori: \textbf{magnitudo}, posizione (\textbf{Latitudine} e \textbf{Longitudine}) e \textbf{data} di un terremoto possano dare in output dei risultati utili all'analisi dei terremoti, dopo aver eseguito opportune operazioni e controlli. Questi risultati vorrei fossero rappresentati su una mappa, in quanto i terremoti stessi si verificano in una area geografica ed \`e quindi pi\`u facile in questo modo analizzare i dati; infatti la rappresentazione su mappa permetter\`a un'analisi dei dati pi\`u semplificata, grazie all'utilizzo di forme geometriche e colori che possono essere sovrapposti alla mappa.\\
L'obiettivo che mi pongo nello sviluppo di questi applicativi si pu\`o facilmente riassumere in due domande:
\begin{itemize}
    \item Quanto \`e rischiosa una determinata area geografica rispetto alle altre?
    \item Quando avverr\`a il prossimo terremoto in una determinata area geografica?
\end{itemize}