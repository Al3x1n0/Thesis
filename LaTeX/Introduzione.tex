\chapter*{Introduzione}
\addcontentsline{toc}{chapter}{Introduzione}

In tutto il mondo ci si pone da anni la domanda: \`E possibile prevedere i terremoti?\\
Ad oggi non \`e ancora possibile rispondervi, ma la maggior parte degli esperti in geologia conviene sul fatto che sia impossibile. Nonostante questa probabile impossibilit\`a rimane uno dei campi di studi pi\`u importanti nel settore dei terremoti. Questo perch\'e i danni causati da un terremoto, se quest'ultimo ha un intensit\`a superiore ad una certa soglia, possono essere catastrofici. Quindi riuscire a darne avviso precedentemente potrebbe non soltanto ridurre i danni, ma salvare molte vite umane.\\
Prendendo le parole della sezione di Napoli che si occupa dello studio dei terremoti: \begin{displayquote}
``\`E possibile, per\`o, attraverso l' individuazione delle aree sismogenetiche, lo studio della loro sismicit\`a storica e recente, dell'assetto tettonico e geologico definire la pericolosit\`a sismica del territorio in base alla quale adottare adeguate misure di prevenzione che possano ridurre gli effetti dei terremoti.''
\end{displayquote}
La pericolosit\`a di cui parla la sezione di Napoli, \`e un'argomento strettamente correlato con la previsione. Questo perch\'e stimare quanto \`e pericolosa una determinata area geografica, ci permette di prestare maggiore attenzione alle tecniche di costruzione degli edifici presenti in quel territorio e di sensibilizzare la popolazione che andr\`a ad abitarli. Questo permetter\`a non soltanto una reazione pi\`u corretta in caso di terremoto da parte delle persone, ma anche un rischio di crollo degli edifici minore.\\
Mentre la previsione \`e ad oggi una domanda aperta, per la pericolosit\`a di una certa area geografica si \`e riusciti a creare delle mappe che con un'opportuna colorazione riportano quanto \`e rischiosa una certa area geografica. Questo \`e possibile attraverso lo studio dei dati storici a disposizione (quindi dei terremoti avvenuti nel passato nell'area geografica presa in considerazione) e anche grazie al costante studio dell'assetto tettonico e geologico del terreno da parte di esperti. Come si potr\`a vedere nel Background di seguito (vedi Figura \ref{img:mappaPericolo}), anche per il nostro territorio nazionale \`e presente una mappa riportante la pericolosit\`a sismica dell'Italia.\\
L'obiettivo che mi pongo \`e di rispondere a due domande:
\begin{itemize}
    \item \textbf{Quanto \`e rischiosa una determinata area geografica rispetto alle altre?}
    \item \textbf{Quando avverr\`a il prossimo terremoto in una determinata area geografica?}
\end{itemize}
Come evidenziato in precedenza i due problemi, pericolosit\`a e previsione, sono strettamente correlati, ma mentre per il primo troviamo una risposta pi\`u semplice, rispondere al secondo \`e decisamente pi\`u complicato.\\
Quello che andr\`o a fare \`e approcciare ai due problemi tenendo in considerazione soltanto i dati storici, facendo quindi un analisi di questi mirata a generare in output dei risultati, rappresentabili su mappa. Per fare ci\`o andr\`o ad utilizzare il linguaggio di programmazione \textit{Python} e due librerie, \textit{Pandas} e \textit{Folium}. La prima libreria (\textit{Pandas}) permette di manipolare database di grandi dimensioni esportati in vari formati (nel mio caso .csv), rendendo le operazioni semplici e veloci. La seconda libreria (\textit{Folium}) permette invece di rappresentare i dati su mappa inserendoci forme geometriche e popup.\\
Per rispondere alle due domande considerer\`o un'area geografica definita da un rettangolo che andr\`o a suddividere in celle. \\
Nel primo caso calcoler\`o la somma delle magnitudo dei terremoti occorsi in ogni cella, successivamente dividendo ogni somma ottenuta per la somma totale avr\`o un vettore normalizzato che somma a 1. Infine colorer\`o la mappa basandomi sul vettore normalizzato.\\
Nel secondo caso estrarr\`o l'intervallo di tempo tra i terremoti con magnitudo maggiore di m occorsi in ogni cella, successivamente calcoler\`o media e varianza della frequenza tra un terremoto ed il successivo per ogni cella ed infine assumendo che l'intervallo di tempo si comporti come una variabile casuale X, stimer\`o con la disuguaglianza di Chebyshev un intervallo di confidenza, che mi permetter\`a di fare una previsione in giorni.