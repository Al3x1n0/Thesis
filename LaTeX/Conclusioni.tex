\chapter{Conclusioni}\label{conclusion}
In questo Capitolo vado a descrivere le conclusioni tratte dal mio lavoro ed a presentare possibili sviluppi futuri.\\
Quello che voglio evidenziare \`e quanto segue:\\
mi sono basato esclusivamente sui dati storici presenti nei cataloghi resi disponibili online. Ho ignorato gli aspetti tecnici, che possono essere i precursori di un terremoto, o anche le scosse di assestamento (che conseguono un forte terremoto) che per essere escluse hanno bisogno di funzioni matematiche, stabilite grazie all'aiuto di geologi, che conoscono il comportamento sismotettonico che avviene in una determinata area dopo un forte terremoto. Nonostrante ci\`o sono riuscito a ricavare una mappa, (vedi Figura \ref{fig:20x20vsWarningMap}) che rispecchia a grandi linee la mappa messa a disposizione dall'INGV. La differenza sostanziale sta nel fatto che per fare analisi di dati come ho fatto io, l'impiego di risorse economiche \`e inferiore a quello di altri tipi di ricerca, quindi apre molte possibilit\`a in quanto l'investimento non richiede una spesa esagerata. Quindi un aspetto molto importante che a volte viene lasciato da parte \`e la creazione di cataloghi concreti, contenenti molte informazioni sui dati storici e costantemente aggiornati. Quello che fa risaltare questa analisi \`e dare maggiore importanza ai cataloghi, ma soprattutto dare la possibilit\`a a chiunque di accedere ai dati attraverso delle queries di ricerca, limitando cos\`i direttamente la base dati che si andr\`a a scaricare.

\section{Sviluppi futuri}

I due programmi che ho sviluppato basati su due approcci differenti, utilizzano ognuno un criterio specifico da me scelto. Questo non preclude la possibilit\`a di andarli a migliorare, magari seguendo suggerimenti di persone qualificate nel campo della geofisica e vulcanologia. Questo per chiarire che soprattutto nel primo approccio mi baso su un criterio prettamente logico, per creare una stima di pericolosit\`a, ma con degli sviluppi successivi il programma potrebbe tirare fuori dei risultati pi\`u esaustivi, soprattutto basandosi su un appoccio pi\`u dettagliato.\\
Nel secondo approccio invece ho utilizzato il calcolo delle probabilit\`a tirando in gioco la formula risultante dalla disuguaglianza di Chebyshev, vedendo gli eventi passati come variabili aleatorie; anche in questo caso vedo delle possibili modifiche che tengono in considerazione metodi pi\`u precisi, magari gi\`a utilizzati in altri campi, che possono andare a modificare l'approccio.\\
Entrambi i programmi quindi sono dinamici in quanto \`e possibile, una volta definita la solida base che ho prontamente preparato, andare ad apportare delle modifiche agli algoritmi interni ai programmi, questo da ai due programmi una potenzialit\`a di crescita futura non indifferente.\\
Parlando invece dei cataloghi utilizzati e come gi\`a detto, se si rispettano le specifiche richieste dai programmi si potr\`a andare ad analizzare qualsiasi catalogo esportato nel giusto formato. Questo permette anche di creare una struttura che si mantiene aggiornata in automatico, ovvero che prenda dinamicamente i dati aggiornati in tempo reale e tracci delle stime sempre pi\`u accurate e aggiornate.\\
Un'altra possibilit\`a che deriva dalla presentazione fatta in precedenza dei programmi, \`e che il rettangolo definito in principio \`e anch'esso preso in input, quindi si potrebbe estendere l'analisi anche ad altre aree geografiche oltre l'Italia.\\
Come si evince gli sviluppi futuri sono molteplici, un altro degno di nota \`e la possibilit\`a di creazione di una interfaccia grafica, che eviti all'utente di interagire direttamente con il terminale passando gli argomenti in input a riga di comando, bens\`i di poter compilare un form, andando ad inserire li tutti gli input necessari per far girare il programma sulla propria richiesta. Naturalmente questa interfaccia pu\`o essere anche creata e messa a disposizione direttamente online, lasciando che i calcoli vengano fatti direttamente dal server WEB ospitante l'applicativo, che a quel punto sarebbe diventato un vero e proprio software pi\`u generale che permette la scelta dell'approccio o ancor di pi\`u ritorna i risultati di entrambi i programmi all'utente.